\documentclass{gshs_thesis}

\graphicspath{{images/}}
% 이곳에 필요한 별도의 패키지들을 적어넣으시오.
%\usepackage{...}
\usepackage{verbatim} % for commment, verbatim environment
\usepackage{spverbatim} % automatic linebreak verbatim environment
\usepackage{listings}
\usepackage{indentfirst}
\usepackage[utf8]{inputenc}
\usepackage{geometry}
\usepackage{algorithm}
\usepackage{algpseudocode}
\usepackage{tikz}
\usepackage{multirow}
\usepackage{pgfplots}
\usepackage{filecontents}
\usepackage{pgfplotstable}
\pgfplotsset{
	compat=newest,
	% label style={font=\scriptsize},
	% ticklabel style={font=\scriptsize},
	% legend style={font=\tiny},
	% major tick length=0.1cm,
	% minor tick length=0.05cm,
	% every x tick/.style={black},
}

\definecolor{mygreen}{rgb}{0,0.6,0}
\definecolor{mygray}{rgb}{0.47,0.47,0.33}
\definecolor{myorange}{rgb}{0.8,0.4,0}
\definecolor{mywhite}{rgb}{0.98,0.98,0.98}
\definecolor{myblue}{rgb}{0.01,0.61,0.98}

\newcommand*{\FormatDigit}[1]{\ttfamily\textcolor{mygreen}{#1}}
%% https://tex.stackexchange.com/questions/32174/listings-package-how-can-i-format-all-numbers
\lstdefinestyle{FormattedNumber}{%
    literate=*{0}{{\FormatDigit{0}}}{1}%
             {1}{{\FormatDigit{1}}}{1}%
             {2}{{\FormatDigit{2}}}{1}%
             {3}{{\FormatDigit{3}}}{1}%
             {4}{{\FormatDigit{4}}}{1}%
             {5}{{\FormatDigit{5}}}{1}%
             {6}{{\FormatDigit{6}}}{1}%
             {7}{{\FormatDigit{7}}}{1}%
             {8}{{\FormatDigit{8}}}{1}%
             {9}{{\FormatDigit{9}}}{1}%
             {.0}{{\FormatDigit{.0}}}{2}% Following is to ensure that only periods
             {.1}{{\FormatDigit{.1}}}{2}% followed by a digit are changed.
             {.2}{{\FormatDigit{.2}}}{2}%
             {.3}{{\FormatDigit{.3}}}{2}%
             {.4}{{\FormatDigit{.4}}}{2}%
             {.5}{{\FormatDigit{.5}}}{2}%
             {.6}{{\FormatDigit{.6}}}{2}%
             {.7}{{\FormatDigit{.7}}}{2}%
             {.8}{{\FormatDigit{.8}}}{2}%
             {.9}{{\FormatDigit{.9}}}{2}%
             %{,}{{\FormatDigit{,}}{1}% depends if you want the "," in color
             {\ }{{ }}{1}% handle the space
             ,%
}


\lstset{%
  backgroundcolor=\color{mywhite},   
  basicstyle=\footnotesize,       
  breakatwhitespace=false,         
  breaklines=true,                 
  captionpos=b,                   
  commentstyle=\color{red},    
  deletekeywords={...},           
  escapeinside={\%*}{*)},          
  extendedchars=true,              
  frame=shadowbox,                    
  keepspaces=true,                 
  keywordstyle=\color{myorange},       
  language=Octave,                
  morekeywords={*,...},            
  numbers=left,                    
  numbersep=5pt,                   
  numberstyle=\tiny\color{mygray}, 
  rulecolor=\color{black},         
  rulesepcolor=\color{myblue},
  showspaces=false,                
  showstringspaces=false,          
  showtabs=false,                  
  stepnumber=2,                    
  stringstyle=\color{myorange},    
  tabsize=2,                       
  title=\lstname,
  emphstyle=\bfseries\color{blue},%  style for emph={} 
}    

%% language specific settings:
\lstdefinestyle{Arduino}{%
    style=FormattedNumber,
    keywords={void},%                 define keywords
    morecomment=[l]{//},%             treat // as comments
    morecomment=[s]{/*}{*/},%         define /* ... */ comments
    emph={HIGH, OUTPUT, LOW},%        keywords to emphasize
}
%% language specific settings:
\lstdefinestyle{Arduino}{%
    style=FormattedNumber,
    keywords={void},%                 define keywords
    morecomment=[l]{//},%             treat // as comments
    morecomment=[s]{/*}{*/},%         define /* ... */ comments
    emph={HIGH, OUTPUT, LOW},%        keywords to emphasize
}
\usepackage{hologo}

% -----------------------------------------------------------------------
%                   이 부분은 수정하지 마시오.
% -----------------------------------------------------------------------
\titleheader{졸업논문청구논문}
\school{과학영재학교 경기과학고등학교}
\approval{위 논문은 과학영재학교 경기과학고등학교 졸업논문으로\\
졸업논문심사위원회에서 심사 통과하였음.}
\chairperson{심사위원장}
\examiner{심사위원}
\apprvsign{(인)}
\korabstract{초 록}
\koracknowledgement{감사의 글}
\korresearches{연 구 활 동}

%: ----------------------------------------------------------------------
%:                  논문 제목과 저자 이름을 입력하시오
% ----------------------------------------------------------------------
% \title{2차원 조파수조의 규칙파 발생 조파기 제작 및 검증} %한글 제목

% \title{Arduino를 이용한 2차원 조파 수조에서의 균형파 발생}
% \title{Teensy 기반 조파기를 이용한 2차원 조파 수조에서의 균형파 발생 및 검증}
%Arduino인가?
\title{모형 실험을 위한 연안 환경 조성 및 규칙파 생성 검증}

% \engtitle{Developing coastal environments for model experiments and verifying balanced wave generation}
\engtitle{Installation of coastal beach slope and verification of regular wave generation in the mini wave flume}

% \engtitle{Generating and Verifying Sinusoidal Wave using Teensy-based Wavemaker in 2-Dimensional Wave Channel}
% \engtitle{Generating Sinusoidal Wave using Arduino in 2-Dimensional Wave Channel}
\korname{남 도 현} %저자 이름을 한글로 입력하시오 (글자 사이 띄어쓰기)
\engname{Nam, Do-Hyun} %저자 이름을 영어로 입력하시오 (family name, personal name)
\chnname{南 都 鉉} %저자 이름을 한자로 입력하시오 (글자 사이 띄어쓰기)
\studid{21033} %학번을 입력하시오

%------------------------------------------------------------------------
%                  심사위원과 논문 승인 날짜를 입력하시오
%------------------------------------------------------------------------
\advisor{Park, Kiehyun}  %지도교사 영문 이름 (family name, personal name)
\judgeone{신 성 원} %심사위원장
\judgetwo{전 영 준}   %심사위원1
\judgethree{박 기 현} %심사위원2(지도교사)
\degreeyear{2024}   %졸업 년도
\degreedate{2023}{6}{30} %논문 승인 날짜 양식

%------------------------------------------------------------------------
%                  논문제출 전 체크리스트를 확인하시오
%------------------------------------------------------------------------
\checklisttitle{[논문제출 전 체크리스트]} %수정하지 마시오
\checklistI{1. 이 논문은 내가 직접 연구하고 작성한 것이다.} %수정하지 마시오
% 이 항목이 사실이라면 다음 줄 앞에 "%"기호 삽입, 다다음 줄 앞의 "%"기호 제거하시오
%\checklistmarkI{$\square$}
\checklistmarkI{$\text{\rlap{$\checkmark$}}\square$}
\checklistII{2. 인용한 모든 자료(책, 논문, 인터넷자료 등)의 인용표시를 바르게 하였다.} %수정하지 마시오
% 이 항목이 사실이라면 다음 줄 앞에 "%"기호 삽입, 다다음 줄 앞의 "%"기호 제거하시오
%\checklistmarkII{$\square$}
\checklistmarkII{$\text{\rlap{$\checkmark$}}\square$}
\checklistIII{3. 인용한 자료의 표현이나 내용을 왜곡하지 않았다.} %수정하지마시오
% 이 항목이 사실이라면 다음 줄 앞에 "%"기호 삽입, 다다음 줄 앞의 "%"기호 제거하시오
%\checklistmarkIII{$\square$}
\checklistmarkIII{$\text{\rlap{$\checkmark$}}\square$}
\checklistIV{4. 정확한 출처제시 없이 다른 사람의 글이나 아이디어를 가져오지 않았다.} %수정하지 마시오
% 이 항목이 사실이라면 다음 줄 앞에 "%"기호 삽입, 다다음 줄 앞의 "%"기호 제거하시오
%\checklistmarkIV{$\square$}
\checklistmarkIV{$\text{\rlap{$\checkmark$}}\square$}
\checklistV{5. 논문 작성 중 도표나 데이터를 조작(위조 혹은 변조)하지 않았다.} %수정하지 마시오
% 이 항목이 사실이라면 다음 줄 앞에 "%"기호 삽입, 다다음 줄 앞의 "%"기호 제거하시오
%\checklistmarkV{$\square$}
\checklistmarkV{$\text{\rlap{$\checkmark$}}\square$}
\checklistVI{6. 다른 친구와 같은 내용의 논문을 제출하지 않았다.} %수정하지 마시오
% 이 항목이 사실이라면 다음 줄 앞에 "%"기호 삽입, 다다음 줄 앞의 "%"기호 제거하시오
%\checklistmarkVI{$\square$}
\checklistmarkVI{$\text{\rlap{$\checkmark$}}\square$}
