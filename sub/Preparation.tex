\section{구두발표}
\begin{quote}
    \begin{center}
    \textbf{지구과학과 졸업논문 구두발표 안내}
        \begin{itemize}
            \item 일시 : 3/27(월) 16:50~18:40 (10분/1인)
            \item 장소: SRC629
            \item 발표자료 :  슬라이드 또는 작성하고 있는 논문
            \item 포함해야 할 내용: 졸업 논문 진행 상황 및 완성 계획
            \item 시간 엄수: 발표시간 7분, 질의응답 3분
        \end{itemize}
            
        성실하게 준비해서 발표하시고, 미비하면 구두발표를 다시 해야 함.
    \end{center}
\end{quote}

\subsection{현황}
\begin{itemize}
    \item \textbf{초록, 감사의 말, 부록, 요약}\\
    마지막에 작성할 예정 (그렇지만 계속 염두에 두고 있음)
    
    \item \textbf{서론}\\
    방학동안 조파기 제작 및 WMT 관련 논문을 다수 찾아봄. WMT와 관련된 여러 논문은 찾았으나 연구 동향과 관련된 논문은 찾기가 어려웠고 현재는 1장 반 정도 작성함. 주 내용은 조파기 관련 연구가 어떻게 발전해왔고 우리 연구는 이미 있던 실험장비를 확장시킨 것이라는 식으로 작성함.    

    \item \textbf{이론적 배경}\\
    WMT와 관련한 여러 논문에서 공통적으로 인용하고 있는 논문이 2개가 있음.

    \begin{quote}
        \begin{enumerate}
            \item Dean and Dalrymple
            \item Hughes
        \end{enumerate}
    \end{quote}

    사실 Hughes의 서적이 더 교과서적이지만 문제는 볼 수가 없다는 것. Dean and Dalrymple의 논문은 piston-type wave maker의 부분만 발췌하여 실었으며 이 외에도 여러 논문에서 본 내용을 바탕으로 이론을 전개했으나 지금 쓴 내용으로는 표절률이 상당히 높을 수 있음. 그리고 수식 전개 내용을 다 넣어야 하는지도 의문이긴 함.

    그래서 WMT는 궁극적으로 파고와 관련된 식을 줌. 그리고 추가적으로 조파수조, 조파기, 소파기 및 파고계에 대한 내용도 실음. 파고계는 이번에 창알이 진행되는 정도를 보아 넣을지 말지를 결정할 예정. 넣지 않아도 큰 문제는 없음.

    \item \textbf{연구방법}\\
    여기가 제일 큰 문제임.
    
    \begin{enumerate}
        \item 수조 확장
        \item 조파기 제작 (Hardware 구성)
        \item 조파기 코딩 (SoftWare 구성)
    \end{enumerate}

    겨울방학 동안 진행한 결과는 조파기 코딩에 멈춰있음. 그래도 최근에는 돌파구를 찾아서 잘 되었으면 좋겠지만 이것도 안 된다고 하면 조파기를 완전히 새로 만들 필요가 있음...

    정말 이 상황은 피하고 싶고 sin형 구동이 된다고 하면 파가 생기는 것을 영상을 찍어 분석할 예정. 분석 방법은 파고계가 잘 작동한다면 적극 활용하겠지만 현재로써는 물에 물감을 풀어 사용하는 것이 그나마 최선임. 그리고 이것도 하나의 모드에 대하여 조파기로부터의 거리에 따라서 실험을 해볼 예정.

    조파기 코딩의 확장으로 디스플레이를 이용한 컨트롤러 완성은 정말정말 이상적으로 모든 실험이 끝난 경우에나 할 수 있을 것 같음. 현재 3D 디자인과 기본 구조는 어느정도 잡혀있으나 이걸 진행할 여유가 없음.

    그리고 조파기가 제대로 작동한다면 컨트롤러 완비와 더불어 연안 모형 관련 실험을 진행할 수 있음. 경사로를 어떻게 할지만 생각해놓으면 저번에 관찰한 쇄파 실험을 할 수도 있고 방파제 조적구조 비교 실험을 간단하게 진행할 수도 있음.

    \textbf{사실 이 모든 것은 이미 작년에도 알고 있던 것이며 조파기가 제대로 작동하면 무엇이든 할 수 있다!}    

    \item \textbf{실험결과 및 결론}
    원래 여기에 예비 실험으로 Arduino에서 모터의 각변위를 출력하여 이런 식으로 움직인다는 것을 보이려고 했으나 문제는 출력되는 값이랑 실제 움직이는 거랑 일치하지 않아보일 때가 많음

    Ex) 모터가 탈조가 나서 돌아가지 않아도 값은 계속 출력됨.

    겨울방학 동안 이 데이터들을 좀 뽑아둔 것이 있는데 어차피 작은 모터로 뽑아둔 것이며 보드만 있어도 충분히 가능한 일이기 때문에 큰 의미가 있나 싶음.
    
\end{itemize}

\subsection{앞으로의 계획}
Teensy 4.0이 왔으니까 이거로 조파기가 돌아가기를 기대해야 함. 모든 것이 순탄하다는 가정하에 졸논이나 과전에 넣을 기본적인 sin형 데이터는 1~2주 안으로 다 뽑을 수 있을 듯함.

확실한 것은 중간 전에 거의 다 끝내야 하는데 중간고사가 끝나면 그 굉장히 좋은 시즌에 여러 행사가 겹쳐있기 때문에 그때 모여서 뭔가를 하는 것은 진짜 쉽지 않음. 기껏해야 중간고사가 끝난 날부터 그 주 목, 금에 조금이나마 모일 수 있을 듯함.
