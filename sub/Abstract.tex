\maketitle  % command to print the title page with the above variables
\setcounter{page}{1}
%---------------------------------------------------------------------

%---------------------------------------------------------------------
\begin{abstracts}     % This creates the heading for the abstract page
	\addcontentsline{toc}{section}{Abstract}  %%% TOC에 표시
	\noindent{
		% A proper experimenting system is required when conducting model experiments, especially in the field of coastal engineering. In this paper, various devices that are used for model experiments were made, and the wavemaker will be verified with its performance of generating sinusoidal waves. The wavemaker is a piston type, and the plate moves on a linear actuator with a stroke of 80$\mathrm{cm}$. It's controlled by the Teensy board.
  
        %       In a short-wave condition, $H/S$ can be calculated analytically, but the relation was not apparent for the wavemaker in the lab, though the linear correlation between $H/S_{msr}$ and $H/S_{thm}$ was shown. Further calibration is required. The devices could be used in other coastal engineering research.
        %%%%%%%%%%%%%%%%%%%%%%%%%%%%%%%%%%%%%%%%%%%%%%%%%%%
        % A robust experimental system is imperative for conducting model experiments, particularly in the domain of coastal engineering. This paper presents the development of various apparatuses utilized in model experiments, specifically focusing on the performance of a piston-type wavemaker in generating sinusoidal waves. The wavemaker incorporates a plate that moves along a linear actuator, providing a stroke length of $80\mathrm{~cm}$, with its control facilitated through the implementation of a Teensy board.

        % Under short-wave conditions, the analytical calculation of the ratio $H/S$ is feasible. However, the relationship was not readily discernible for the wavemaker within the laboratory setting, despite the manifestation of a linear correlation between $H/S_{msr}$ and $H/S_{thm}$. Consequently, additional calibration is deemed necessary. Notably, these devised apparatuses hold potential for utilization in other realms of coastal engineering research

        A robust experimental system is imperative for conducting model experiments, particularly in coastal engineering. This study presents the development of various equipment utilized in model experiments, specifically focusing on the performance of a piston-type wavemaker in generating sinusoidal waves. The wavemaker incorporates a plate that moves along a linear actuator, providing a stroke length of 80$\mathrm{~cm}$, with its control facilitated by implementing a Teensy board. Under short-wave conditions, the analytical calculation of the ratio $H/S$ is feasible. However, the relationship could have been more readily discernible for the wavemaker within the laboratory setting, despite a linear correlation between $H/S_{msr} and H/S_{thm}$. Consequently, additional calibration is deemed necessary. Notably, these equipment and instrument are potentially utilized in other realms of coastal engineering research.
  
	}
\end{abstracts}

%---------------------------------------------------------------------

%---------------------------------------------------------------------
% \begin{abstractskor}        %this creates the heading for the abstract page
% 	\addcontentsline{toc}{section}{초록}  %%% TOC에 표시
% 	\noindent{
% 		초록(요약문)은 가장 마지막에 작성한다. 연구한 내용, 즉 본론부터 
% 		요약한다. 서론 요약은 하지 않는다. 대개 첫 문장은 연구 주제 
% 		(+방법을 핵심적으로 나타낼 수 있는 문구: 실험적으로, 
% 		이론적으로, 시뮬레이션을 통해)를 쓴다. 다음으로 연구 방법을 
% 		요약한다. 선행 연구들과 구별되는 특징을 중심으로 쓴다. 뚜렷한 
% 		특징이 없다면 연구방법은 안써도 상관없다. 다음으로 연구 결과를 
% 		쓴다. 연구 결과는 추론을 담지 않고, 객관적으로 서술한다. 

A robust experimental system is imperative for conducting model experiments, particularly in coastal engineering. This study presents the development of various equipment utilized in model experiments, specifically focusing on the performance of a piston-type wavemaker in generating sinusoidal waves. The wavemaker incorporates a plate that moves along a linear actuator, providing a stroke length of 80 cm, with its control facilitated by implementing a Teensy board. Under short-wave conditions, the analytical calculation of the ratio H/S is feasible. However, the relationship could have been more readily discernible for the wavemaker within the laboratory setting, despite a linear correlation between H/Smsr and H/Sthm. Consequently, additional calibration is deemed necessary. Notably, these equipment and instrument are potentially utilized in other realms of coastal engineering research.
% 		마지막으로 결론을 쓴다. 이 연구를 통해 주장하고자 하는 바를 
% 		간략히 쓴다. 요약문 전체에서 연구 결과와 결론이 차지하는 비율이 
% 		절반이 넘도록 한다. 읽는 이가 요약문으로부터 얻으려는 정보는 
% 		연구 결과와 결론이기 때문이다. 연구 결과만 레포트하는 논문인 
% 		경우, 결론을 쓰지 않는 경우도 있다.
% 	}
% \end{abstractskor}


%----------------------------------------------
%   Table of Contents (자동 작성됨)
%----------------------------------------------
\cleardoublepage
\addcontentsline{toc}{section}{Contents}
\setcounter{secnumdepth}{3} % organisational level that receives a numbers
\setcounter{tocdepth}{3}    % print table of contents for level 3
\baselineskip=2.2em
\tableofcontents


%----------------------------------------------
%     List of Figures/Tables (자동 작성됨)
%----------------------------------------------
\cleardoublepage
\clearpage
\listoftables
% 표 목록과 캡션을 출력한다. 만약 논문에 표가 없다면 이 위 줄의 맨 앞에 
% `%' 기호를 넣어서 주석 처리한다.

\cleardoublepage
\clearpage
\listoffigures
% 그림 목록과 캡션을 출력한다. 만약 논문에 그림이 없다면 이 위 줄의 맨 앞에 
% `%' 기호를 넣어서 주석 처리한다.

\cleardoublepage
\clearpage
\renewcommand{\thepage}{\arabic{page}}
\setcounter{page}{1}