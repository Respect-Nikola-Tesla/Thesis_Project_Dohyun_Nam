%-----------------------------------------------------
%   감사의 글
%-----------------------------------------------------
\begin{acknowledgements}
\addcontentsline{toc}{section}{Acknowledgements}  %%% TOC에 표시
    
    % First and foremost, I would like to thank my research supervisor, Kihyuen Park. Without his assistance and dedication, this paper couldn't be accomplished. He helped me handle electronic devices, design circuits, and PCB, give ideas for our experimenting system, and even write documents with \LaTeX. I would like to thank him very much for his great support over the past year. Also, I would like to appreciate Youngjoon Jeon, an advisor to the research for comparing the masonry structure of the breakwater and making a proper environment for it in 2021. From this research, he made a wave channel with a length of $4,000\mathrm{~mm}$, the one already built before this research. If it had not been for him and his hard work, this research could not have gotten started.

    % I would also like to show gratitude to my consultant, professor Sungwon Shin at ERICA Campus, Hanyang University. He advised us and introduced the basic concept of coastal engineering. Also, he had invited our team to his lab, and even came to our lab himself, giving out feedback on our experimenting system. The basic design of our devices was motivated by his lab, and he willingly let us do so, plus introducing related studies and experiments.
    

    % During the whole research, I had my teammates, Hyosong Baek, and Sihyun Lee, who communicated and solved problems altogether. Sihyeon helped to make wave absorbers and a wave gauge. She didn't resist giving out ideas when we faced many structural problems, and most of the ideas eventually worked out finely. Hyosong is our best programmer, coding most of the commands for our Wavemaker's, software. There had been a lot of problems with various parameters, and she discovered what \textit{MaxSpeed} and \textit{MaxAcceleration} do, and how those parameters eventually affect the signal. \textit{Code Test} involves a lot more trial and error, and Hyosong was the one who was in charge of it. She connected a button to the circuit so that it works as a remote control, and is currently working on its manual.

    % Though not every moment was hopeful and successful and even sometimes, our wave maker failed to work, I could keep on the research because so many people gave their hands to me. They tried to give ideas, hope, and confidence. I feel grateful for all of them.

    First and foremost, I extend my deepest appreciation to my research supervisor, Kihyuen Park, whose exceptional guidance and unwavering dedication have been pivotal to the successful completion of this paper. His expertise encompassed various aspects, including the handling of electronic devices, circuit and PCB design, the conceptualization of the experimenting system, and proficiency in document preparation using \LaTeX. I am sincerely grateful for his invaluable support throughout the past year.

    I would also like to acknowledge the invaluable contributions of Youngjoon Jeon, an advisor to this research, who played a crucial role in comparing the masonry structure of the breakwater and establishing an optimal experimental environment in 2021. Through his diligent efforts, a wave channel measuring 4,000 mm in length was constructed, providing a solid foundation for this research. Without his hard work and dedication, this research endeavor would not have been initiated.

    Additionally, I am indebted to Professor Sungwon Shin, my consultant at ERICA Campus, Hanyang University, for his expert guidance and introduction to the fundamental concepts of coastal engineering. Professor Shin graciously welcomed our team into his laboratory, provided invaluable feedback on our experimenting system, and shared relevant studies and experiments. The basic design of our devices drew inspiration from his laboratory, and his unwavering support allowed us to proceed with our research.

    Throughout the entire research process, I was fortunate to have the collaboration and support of my teammates, Hyosong Baek and Sihyun Lee, who greatly contributed to effective communication and joint problem-solving. Sihyeon played a crucial role in constructing wave absorbers and wave gauges. Her proactive approach to offering ideas to overcome numerous structural challenges proved highly effective. Hyosong, our skilled programmer, demonstrated exceptional coding proficiency in developing the software for our wavemaker. In the face of various parameter-related challenges, she successfully deciphered the functions and implications of critical parameters such as \textit{MaxSpeed} and \textit{MaxAcceleration}, ultimately influencing the signal quality. The \textit{Code Test} phase involved extensive trial and error, for which Hyosong assumed responsibility. Additionally, she incorporated a remote control functionality by integrating a button into the circuit, and she is currently working on its manual.

    While not every moment of this research journey was marked by success and optimism, and despite occasional setbacks where our wavemaker failed to operate as intended, I was able to persevere due to the generous support of numerous individuals. Their continuous provision of ideas, hope, and confidence greatly uplifted my spirits, for which I am deeply grateful.
    
\end{acknowledgements}

%-----------------------------------------------------
%   연구활동 
%-----------------------------------------------------
\begin{researches}
% \addcontentsline{toc}{section}{연구활동}  %%% TOC에 표시
% \begin{itemize}
% \item{2011학년도 교내 R\&E 발표대회에서 장려상 수상}
% \item{2012학년도 교내 R\&E 발표대회에서 장려상 수상}
% \item{2013학년도 교내 R\&E 발표대회에서 장려상 수상}
% \item{2014학년도 교내 R\&E 발표대회에서 장려상 수상}
% \item{2015학년도 교내 R\&E 발표대회에서 장려상 수상}
% \item{2016학년도 교내 R\&E 발표대회에서 장려상 수상}
% \item{2017학년도 교내 R\&E 발표대회에서 장려상 수상}
% \item{2018학년도 교내 R\&E 발표대회에서 장려상 수상}
% \item{2019년 노벨 물리학상 수상}
% \end{itemize}

\addcontentsline{toc}{section}{Perfomance of the Research}  %%% TOC에 표시
\begin{itemize}
\item{2022학년도 추계 지구과학학술대회에서 장려상 수상}
\end{itemize}
\end{researches}