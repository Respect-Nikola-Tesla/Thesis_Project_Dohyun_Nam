%Text Check!
\section{Introduction}

%% Notes taken during reading references


%A combined physical and numerical modeling methodology was employed.
%반사파 소멸은 정확도에 아주 중요한 요소.

In the fields of marine engineering, coastal engineering, hydrodynamics, and other fluid-related branches, the behavior of fluids holds paramount importance as it provides crucial insights into related variables and further information, including velocity fields, mass distributions, and pressure fields. These fundamental variables are essential for calculating factors such as object forces, object durability, particle behavior, and the energy resulting from fluid flow impact. To study fluid behavior, two primary methodologies are commonly employed: simulation and experimental processing. Each approach possesses distinct strengths and weaknesses.

As the theoretical understanding of hydrodynamics advanced, simulation models became more concrete and precise, giving rise to Computational Fluid Dynamics (CFD) as a systematic simulation method. CFD offers numerous turbulence computational models, such as the $k-\epsilon$ model or the $k-\omega$ model \cite{sodja2007turbulence}. Users can adjust time and accuracy by modifying time step durations and mesh sizes. While CFD is widely used in laboratory experiments and corporate modeling tests, it is fundamentally limited as a simulation tool.

However, conducting experiments on life-sized objects such as vessels, coastal terrain, breakwaters, or wave power plants is nearly impossible due to economic and environmental constraints. In such cases, model experiments serve as a popular alternative, allowing for the modification of relevant factors. Many previous studies have utilized model experiments, often in conjunction with simulations, to gain insights into various factors \cite{izquierdo2019experimental}. Model experiments require the use of devices and environments capable of replicating open sea or coastal conditions. Wave channels, wavemakers, wave absorbers, wave gauges, and other related equipment are commonly employed.

In the context of coastal engineering, several laboratory studies have been conducted, including comparisons of breakwater structures, requiring the creation of suitable environments with appropriate devices: wave channels and wavemakers. However, obtaining these two fundamental pieces of equipment poses challenges as they are difficult to acquire and often come with a significant price tag. (Occasionally, manufacturers targeting enterprises offer large, expensive products.) In 2021, a small wave channel with dimensions of $4,000\mathrm{~mm} \times 400\mathrm{~mm} \times 300\mathrm{~mm}$ was constructed. Additionally, a wavemaker was built; however, it possessed several limitations, lacking controllable parameters. While the wavemaker successfully generated sinusoidal waves, its inability to offer parameter control hindered its effectiveness.

The primary objective of this research is to successfully generate various waves using proper wavemakers and other related devices \cite{o2017methods}. By addressing these challenges, this study aims to contribute to the advancement of coastal engineering research.

% In the fields of marine engineering, coastal engineering, hydrodynamics, and other fluid-related branches, the behavior of the fluid is the most important factor since related variables and further information could be derived, such as velocity field, mass distribution, pressure field, etc. These basic variables should be known to calculate further, such as the force exerted on the object, the durability of the object or the behavior of particles, and even energy caused by the impact of the flow of the fluid. The methodology provided is using simulation or processing experimentation. Both methods have their strengths and weaknesses.

% As hydrodynamics developed theoretically, simulation models became concrete and precise, and Computational Fluid Dynamics(CFD) emerged, systematically changing the simulation method. There are lots of turbulence computational models that the program provides, such as the $k - \epsilon$ model or the $k - \omega$ model \cite{sodja2007turbulence}. Users can regulate the time and accuracy by modifying the duration of the time step and the size of the mesh. Even though CFD is used widely in lab experiments and corporations' modeling tests, it has a critical limitation in that it's just a simulation.

% But, it's almost impossible to conduct various experiments with life-sized objects, such as vessels, coastal terrain, breakwater, wave power plant, etc. It's economically and environmentally limited. For an alternative, model experiments are used popularly, with proper factors being modified. Most of the previous studies involving experiments use model experiments. Also, simulation can give some clues about the factor, and several previous studies used both methodologies in parallel \cite{izquierdo2019experimental}. Model experiments require devices and environments that can reproduce what happens at open sea or on the coast. Wave channels, wavemakers, wave absorbers, wave gauges, and other related devices are used.

% Several studies related to coastal engineering had been conducted in the lab, such as comparing the structure of breakwaters, and students needed proper devices to create a suitable environment: Wave channel and wavemaker. These two fundamental pieces of equipment are difficult to obtain and they are quite expensive (Occasionally, companies that make these devices target enterprises, so products are usually big and expensive). In 2021, a small wave channel was constructed. Its dimension is $4,000\mathrm{~mm} \times 400\mathrm{~mm} \times 300 \mathrm{~mm}$. Also, there was a wavemaker built, but it had too many limitations. Though it successfully generated several sinusoidal waves, there were no parameters that we can control.

% This research is aimed to generate various waves successfully, with proper wavemakers, along with other related devices \cite{o2017methods}.

%%%%%%%%%%%%%%%%%%%%%%%%%%%%%%%%%%%%%%%%%%%%%%%%%%%%%%%
% ver 1.
% Fluid-related factors play a crucial role in the fields of ship engineering, coastal engineering, hydrodynamics, and other branches of fluid mechanics. Understanding the behavior of fluids and the forces they exert is of paramount importance. By studying relevant variables, such as the velocity field, valuable information can be derived, including the forces and pressures exerted on objects submerged in the fluid. These fundamental variables are essential for calculating factors such as object durability, particle behavior, and even the energy generated by fluid flow impact. Researchers typically rely on two methodologies for obtaining this information: simulation and direct experimentation, each with its strengths and weaknesses.

% The advancement of theoretical hydrodynamics has led to the development of concrete and precise simulation models, giving rise to Computational Fluid Dynamics (CFD). This systematic approach has significantly transformed the simulation methodology. CFD software offers various turbulence computational models, such as the k-epsilon model or the k-omega model. Users have the flexibility to adjust parameters like time step duration and mesh size to control the accuracy and temporal resolution of simulations. Although CFD is extensively utilized in laboratory experiments and industrial modeling tests, it has a critical limitation: the simulation results may differ from real-life experiments.

% Conducting experiments with full-scale objects, such as vessels, coastal terrains, breakwaters, or wave power plants, is nearly impossible due to economic and environmental constraints. As an alternative, model experiments are widely employed, wherein specific factors are appropriately scaled down. Previous studies have predominantly relied on model experiments, while some researchers have utilized both simulation and experimentation in parallel to gain complementary insights. Model experiments require specialized equipment and environments capable of replicating the conditions experienced in open seas or coastal regions. Wave channels, wave makers, wave absorbers, wave gauges, and other related devices are commonly employed in these experiments.

% Several studies in the field of coastal engineering have been conducted in laboratory settings, including investigations on breakwater structures. Creating an appropriate experimental environment necessitates the use of essential devices: a wave channel and a wave maker. However, acquiring these two fundamental pieces of equipment can be challenging and costly, as the manufacturers of such devices often target commercial enterprises, resulting in large and expensive products. In 2021, a small-scale wave channel with dimensions of 4 m x 40 cm x 30 cm was constructed, but the accompanying wave maker had several limitations. Although it successfully generated sinusoidal waves, it could not control various parameters.

% The primary objective of this research is to develop an effective wave maker, along with other related devices, to successfully generate a wide range of wave types for experimental purposes.