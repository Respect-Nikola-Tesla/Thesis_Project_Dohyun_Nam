\begin{summary}
\addcontentsline{toc}{section}{Summary}  %%% TOC에 표시
% 37기부터는 영어로 졸업논문을 작성한 학생은 반드시 5페이지 내외의 한글 요약문을 작성해야 합니다. 한글로 작성하는 학생은 이 부분을 작성하지 않아도 됩니다. gs19xxx.tex에서 주석 처리하십시오.
    본 연구는 교내에서 진행되는 다양한 연안 공학 연구에 활용될 수 있는 환경을 조성하는데 목적이 있다. 여러 연안 공학 관련 연구가 진행됨에 따라 그 필요성이 대두되었고 2021년 규격 $4,000\mathrm{~mm}\times300\mathrm{~mm}\times400\mathrm{~mm}$의 조파수조 및 조파기가 제작되었다. 그러나 제작된 조파기는 매개변수에 의해서 조절이 가능한 것이 아닌 버튼에 의해 제한적으로 수 개의 파만 생성할 수 있었다. 매개변수를 도입하여 맞춤형 파를 생성할 수 있도록 하는 조파기를 제작하고 조파수조에 연안 모형, 소파기 등을 추가적으로 설치하여 연안 모형 실험에 이용될 수 있는 환경을 조성하는 것이 본 연구의 목표이다.

    본 연구에 필요한 장비는 조파수조, 조파기, 연안 모형, 소파기로 나눌 수 있다. 조파기는 판의 운동 방식에 따라 다양한 종류로 나뉘며 본 연구에서는 판의 모든 요소가 수평적으로 움직이는 피스톤형을 활용한다. 피스톤형 조파기는 2차원 파만 생성할 수 있으며 기존의 조파기는 FUYU사의 리니어 액추에이터와 컨트롤러를 기반으로 제작되었으나 다양한 파를 생성하기에는 제약이 있다.

    조파기에 의해 생성된 파는 선형 조파이론을 통해 해석적으로 접근할 수 있다. 조파이론의 궁극적인 목표는 수면의 형태, 즉 파의 형태를 알아내는 것이며 유체가 만족하는 여러 방정식과 경계조건 및 근사를 통해 구한다. 본 연구에서는 Dean and Darylmple의 이론을 차용하며 그 결과 판의 이동범위인 스트로크($S_{0}$)와 파고 $H$ 사이의 관계를 얻을 수 있다(식 \ref{eq_023}). 하지만 이는 판이 sin형으로 움직이는 경우 심해파(short wave)에 대한 근사이며 천해파 혹은 반사파를 고려하는 경우 다른 이론을 적용해야 한다.

    소파기는 크게 능동형과 수동형으로 나뉜다. 능동형 소파기는 파가 벽과 충돌하는 지점에 새로운 조파기를 두어 상쇄파를 발생시키거나 애초에 조파기가 파를 발생시킬 때 반사파를 분석하여 생성파를 수정하는 방식이 흡수 조파 방식으로 작동한다. 반사계수를 효과적으로 줄일 수 있으나 고가이며 고도의 기술을 필요로 하고 적용할 수 있는 환경이 제한적이다. 반면에 수동형 소파기는 대중적으로 쓰이며 파가 벽이나 장애물과 충돌할 때 그 에너지를 최대한 소멸시킨다. 주로 다공성 구조로 제작되며 기존의 소파기 또한 3D 프린터로 제작한 다공성 구조의 직육면체 형태였다.

    모형실험을 하는 경우에는 실제 실험계를 축소시키기 때문에 여러 물리적 특성이 바뀐다. 이를 보정해주어야 하는데 Froude 상사 법칙이 적용되어야 한다. 이는 무차원 수인 Froude 수가 보존되어야 한다는 것으로 Reynolds 수, Prantdl 수 등 다양한 무차원 수가 있으나 본 논문에서는 유체의 점성이나 확산보다는 파의 전달에 관한 것으로 Froude 수가 보존되어야 한다. 이는 식 \ref{froude number}과 같이 적용되며 바다와 실험계는 모두 파가 움직이므로 분산관계식을 대입하여 새로 쓰면 식 \ref{froude number equation}이 성립해야 한다. 이를 통해서 천해파는 $\omega \sim \pi$, 심해파는 $\omega \sim 4\pi$의 수준이 되어야함을 알 수 있다.

    이론적으로 실험계에 대한 분석이 끝난 후 조파수조 연장, 조파기 및 소파기, 연안모형 제작을 시작하였다. 본교에 있는 조파수조는 2차원 조파수조로 수조 내부 유체를 비회전성 유체로 간주하며 길이 $2,000\mathrm{~mm}$의 모듈 2개가 이어져 있는 형태이다. 이 사이에 새로 제작한 모듈 하나를 끼워넣어 총 $6,000\mathrm{~mm}$ 길이의 수조를 완성하였다 (그림 \ref{wave channel}). 수조가 길어지면 조파기, 연안 모형, 소파기 등을 놓을 수 있는 공간이 늘어나며 반사파에 의한 간섭효과를 줄일 수 있다. 각 모듈은 알루미늄 프로파일 3030 시리즈로 뼈대가 제작되었으며 아크릴 판으로 벽면을 막았다. 물이 새지 않도록 실리콘 패드를 접합부에 끼워넣었고 옆에 아크릴 조인트를 붙여 압착되도록 하였다. 추가적으로 실리콘 마감을 하여 물이 새지 않도록 하였다 (그림 \ref{connection}).
    
    연안 모형은 경사로를 설치하였다. 기울기가 1/10이도록 아크릴 판을 설치하였으며 알루미늄 프로파일로 제작한 교각을 3개의 단으로 나누어 설치하였고 탈부착이 가능하여 다양한 실험에 사용할 수 있다. 이 위에 연안 모형이나 파력 발전소 모형 등을 놓아 실험할 수 있으며 조파기의 성능을 검증하는 실험에서는 사용하지 않을 것이다. 또, 소파기를 구멍이 뚫린 바구니에 수세미를 집어넣어 쌓은 다공성 구조의 직육면체를 활용하였다.
    
    조파기는 크게 구동부와 제어부로 나뉜다. 구동부는 $80\mathrm{~cm}$ 길이의 리니어 액추에이터와 이를 지탱하는 프레임으로 구성된다. 리니어 액추에이터는 모터가 돌아가면서 그에 달린 나사산을 통해 이동부가 앞뒤로 운동하여 모터의 회전 운동을 병진운동으로 바꿔주는 장치이다. 최대 $40\mathrm{~N}$의 힘을 낼 수 있으며 관련 정보는 표 \ref{specifiacion of linear actuator}에 제공되어 있다. 제어부는 MCU 역할을 하는 Teensy3.2, 모터를 직접적으로 제어하는 모터 드라이버 DRV8825, ST-M5045, 그리고 정격전압을 제공해주는 변압기로 구성된다. Teensy3.2는 여타 아두이노 보드에 비해 통신 속도가 빠르고 사양이 좋다는 것이 장점이며 구체적인 스펙은 표 \ref{Specification of Teensy 3.2}에 제시되어 있다. Teensyduino를 통해 코딩하여 명령을 내릴 수 있다. DRV8825와 ST-M5045는 모터 드라이버이며 스텝 모터를 제어한다. 구체적인 스펙은 표 \ref{Specification of DRV8825}, \ref{Specification of ST-M5045}에 제시되어 있으며 DRV8825로도 리니어 액추에이터의 모터를 돌릴 수 있으나 MaxSpeed, MaxAcceleration 등의 매개변수에 따라서 탈조가 나기 때문에 더 활용범위가 넓은 ST-M5045를 사용하였다. 회로의 전압 및 전류에 따라서 허용되는 최대 속도, 가속도가 다르며 외장형 모터 드라이버인 ST-M5045는 외부의 스위치로 전류와 스텝 수를 조절할 수 있다. 변압기를 통해 $36\mathrm{~V}$가 회로에 공급되고 있으나 필요에 따라 $48\mathrm{~V}$로 올릴 수 있다.

    생성된 파를 측정하려면 파고 데이터를 얻을 수 있는 장치가 필요하다. 실험실에는 파고계가 없으며 이를 대체하기 위해 물에 뜨는 작은 스티로폼 조각을 실에 끼워 수평적으로 고정하여 파의 개형에 따라 부표가 위 아래로 운동하도록 하였다. 이를 녹화하여 영상을 분석해 파의 개형을 알아냈다.

    조파기는 코드를 짜서 구동할 수 있으며 기본적으로 주어지는 여러 라이브러리로는 모터의 속도가 상당히 느려 Teensystep을 활용하였다. 이를 통해 각변위를 직접 대입하는 방식으로 모터의 회전 운동을 제어할 수 있다. sin파를 생성하는 것을 목표로 하며 모터의 각변위를 sin형으로 두면 여러 매개변수가 존재하며 각변위를 업데이트하는 시간 $N$을 제외하면 진폭 $A$, 각진동수 $\omega$가 주요한 매개변수이다. 시험 구동에 의하면 모터는 입력한 sin형 신호와 $\omega$가 동일했으나 $\omega$가 증가할수록 $A$가 작아진다는 특징이 있었다. 이는 모터의 한계에 의한 것으로 간주되며 임시적으로 공급 전압과 전류를 높였으나 막기는 어려울 것이다. 즉, 본 실험에서는 입력한 신호와 판의 움직임, 생성되는 파 3가지를 관찰하고 각 파의 $\omega$, $A$를 비교해야 한다. 실험계는 그림 \ref{Experimnet_System}와 같이 조성되었다.

    실험 결과 $\omega$는 3개의 파 모두 동일했으나 $S_0/2$와 $A$는 일치하지 않았고 $\omega$가 커짐에 따라 $S_{0}$가 감소하였다. 또, 분산관계식과 구조적인 값, 앞선 이론에 의한 $H/S$와 측정한 $H/S$를 비교한 결과 (그림 \ref{H/S graph}) $\omega$가 큰 구간($H/S$가 작은 구간)에서 기울기가 1인 선형 관계를 볼 것으로 예상했으나 실제는 기울기가 약 0.69인 선형 관계를 관찰할 수 있었다. 이는 주어진 이론이 실험 상황과 잘 맞지 않으며 근사를 덜 적용하여 2차까지 풀어낸 이론이나 비선형 조파이론 등 다른 이론을 적용해야 할 것이다. 또, 조파기는 실험장비이므로 적절한 calibration이 필요하며 더 많은 실험을 진행해야 한다.

    결론적으로, calibration을 진행한다면 조파기로 원하는 형태의 sin파를 만들 수 있다. 조파기 구동코드의 여러 매개변수를 바꾸면서 실험을 진행하면 더 나은 이해를 할 수 있을 것이다. 또, 소파기의 성능 검증이나 진보된 파고계를 활용하여 실험계의 질을 높일 수도 있으며 이를 통해 다양한 연안 공학 관련 실험을 진행할 수 있다.

\end{summary}
 